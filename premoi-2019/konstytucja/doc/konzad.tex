\author{Jonasz Aleszkiewicz}

\documentclass{sinol}
\usepackage{microtype}

\title{Konstytucja}
\RAM{128}
\date{2019-03-02}
\konkurs{PreMOI}
\id{kon}

\begin{document}
\begin{tasktext}

Po upadku zbrodniczego reżimu, wszystkie siły starego i nowego porządku zebrały się, aby ułożyć  konstytucję dla nowo powstałego ustroju. Oparta na ideałach demokracji, ma być najważniejszym dokumentem w państwie.  Dlatego zarówno stary jak i nowy porządek bardzo chcą mieć czym się pochwalić po ukończonych negocjacjach.

Nie wiadomo jednak dokładnie czego żadna ze stron chce. Jedyne co można wywnioskować z ich ogólnikowych populistycznych wypowiedzi to to, że ma to jakiś związek z najmniejszymi i największymi leksykograficznie słowami.

Służba cywilna postanowiła pomóc w negocjacjach poprzez napisanie programu znajdującego dla każdego wiersza konstytucji najmniejsze i największe leksykograficznie słowo.  Niestety, od czasu zamachu na Ministra Finansów niemożliwe jest przydzielenie na to zadanie budżetu.  Dlatego teraz Ty jesteś jedyną nadzieją narodu -- napisz taki program.

\medskip

\section{Wejście}

Na wejściu podana jest dowolna ilość linii, każda z nich odpowiadająca jednemu wierszowi konstytucji. Każdy wiersz składa się ze słów, znaków interpunkcyjnych ze zbioru \texttt{?!.;'"\&()} oraz liczb. Słowo definiujemy jako spójny ciąg małych lub dużych liter alfabetu łacińskiego o maksymalnej możliwej długości.  Zauważ, że w szczególności między słowami może być tylko znak interpunkcyjny, np. \texttt{K\&R C} to trzy słowa. Ostatni wiersz wejścia jest pusty.

Sumaryczna liczba znaków na wejściu nie przekroczy $10^6$.

\medskip

\section{Wyjście}

Dla każdej linii wejścia, program powinien wypisać najmniejsze i największe leksykograficznie słowo jakie pojawiło się w danej linii.  Przy porównywaniu pomijamy wielkość liter: nie ma tak, że dzielimy je na małe i~duże.  Jeżeli jedno słowo jest prefiksem drugiego, jest od niego mniejsze leksykograficznie, np. \texttt{Java} $<$ \texttt{JavaScript}. Jeżeli kilka słów ma taką samą kolejność leksykograficzną, program powinien wypisać to, które wystąpiło najwcześniej w danej linii.

\medskip

\section{Przykłady}

\makeexample{0a}

% Wierząc, że IV Rzeczpospolita powstanie
% Z realnych planów i pięknych snów,
% Chcemy wypełnić konsekwentnie swe zadanie
% Znając znaczenie prostych słów:
%
% Prawda. Wiara. Żarliwość.
% Piękno. Praca. Ofiarność.
% Prawo i Sprawiedliwość.
% Polska i Solidarność.
\makeexample{0b}

\makeexample{0c}
\textit{Wyjaśnienie do przykładu:} słowa \texttt{Konstytucja} i \texttt{konstytucja} są równe leksykograficznie przy pomijaniu wielkości liter, więc wynikiem jest to z nich, które wcześniej wystąpiło w danej linii.

\end{tasktext}
\end{document}
