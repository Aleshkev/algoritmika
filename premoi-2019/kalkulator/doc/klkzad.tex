\author{Jonasz Aleszkiewicz}

\documentclass{sinol}
\usepackage[T1]{polski}
\usepackage{microtype}

\title{Kalkulator}
\RAM{128}
\date{2019-03-02}
\konkurs{PreMOI}
\id{klk}

\begin{document}
\begin{tasktext}

Bajtek wymyślił innowacyjną nową operację między dwoma liczbami: $a \otimes b = 2a + b$. Łączy się ona w lewo, czyli $a \otimes b \otimes c = (a \otimes b) \otimes c$.

Bajtek nie zna jeszcze dokładnych zastosowań, ale chciałby już teraz mieć program, który policzy wynik dla dowolnego wyrażenia składającego się z dowolnie dużych nieujemnych liczb, nawiasów i jego operacji. Ponieważ C++ nie przewidział $\otimes$, na wejściu ten operator zostanie przedstawiony jako \texttt{\#}.

Ponieważ wynik może być duży, a Bajtek nie lubi czytać dużych liczb, wystarczy mu wynik modulo $10^9 + 7$.

\medskip

\section{Wejście}

Na jedynej linii wejścia jest poprawne wyrażenie, którego wynik interesuje Bajtka. Wszystkie spacje zawsze mogą być pominięte. Długość linii (licząc ze spacjami) nie przekroczy $2 \cdot 10^6$.

\medskip

\section{Przykłady}

\makeexample{0a}
\textit{Wyjaśnienie:} $1 \otimes 1 = 2(1) + 1 = 3$

\makeexample{0b}
\textit{Wyjaśnienie:} $1 \otimes 1 \otimes 1 \otimes 1 \otimes = ((1 \otimes 1) \otimes 1) \otimes 1 = (3 \otimes 1) \otimes 1 = 7 \otimes 1 = 15$

\makeexample{0c}
\textit{Wyjaśnienie:} $(1 \otimes 1) \otimes (1 \otimes 1) = 3 \otimes 3 = 9$

\makeexample{0d}
\end{tasktext}
\end{document}
